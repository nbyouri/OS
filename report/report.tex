\documentclass{report}

\usepackage[utf8]{inputenc}
\usepackage[T1]{fontenc}
\usepackage[francais]{babel}
\usepackage{listings}
\usepackage[a4paper]{geometry}
\usepackage{graphicx}
\usepackage[export]{adjustbox}
\usepackage{titlesec}
\usepackage{color}
\usepackage[toc, page]{appendix}

\definecolor{xcodekw}{rgb}{0.75, 0.22, 0.60}
\definecolor{xcodestr}{rgb}{0.89, 0.27, 0.30}
\definecolor{xcodecmt}{rgb}{0.31, 0.73, 0.35}

\titleformat{\chapter}[display]
  {\centering\normalfont\huge\bfseries}
  {\chaptertitlename\ \thechapter}
  {20pt}
  {\Huge}

\geometry{hscale=0.75,vscale=0.85,centering}

\DeclareUnicodeCharacter{00A0}{ }

\lstset{
  frame=tb,
  language=C,
  aboveskip=3mm,
  belowskip=3mm,
  showstringspaces=false,
  columns=flexible,
  basicstyle={\small\ttfamily},
  numbers=none,
  breaklines=true,
  breakatwhitespace=true,
  tabsize=3,
  keywordstyle=\color{xcodekw},
  stringstyle=\color{xcodestr},
  commentstyle=\color{xcodecmt}
}

\renewcommand{\thesection}{\arabic{section}}
\renewcommand\appendixtocname{Annexes}
\renewcommand\appendixname{Annexes}
\renewcommand\appendixpagename{Annexes}

\title{Projet OS}
\author{Youri \bsc{Mouton}, Rémy \bsc{Voet} et Samuel \bsc{Monroe}}
\date{Janvier 2015}

\begin{document}

\maketitle

\chapter{Introduction}

	\section{Introduction}

		Le projet qui fait l'objet de ce rapport est un travail de programmation
		en C système sous UNIX, dans le but d'exploiter de manière pratique les connaissances théoriques acquises au cours de Mme. Masson.
		\newline

		Ce travail consiste en l'écriture de trois programmes distincts autour d'une thématique aéronautique, échangeant des informations entre eux via divers moyens, ceci tout en gérant les éventuels conflits et erreurs liés à ces échanges.\newline
		Ces trois programmes sont :\newline
		\begin{description}
			\item[Tour de controle : ]
			Fait office de \textbf{serveur}, joue le rôle majeur en s'occupant de recevoir les demandes des pilotes, en allant chercher les informations météo fournies par le centre météo, et en renvoyant celles-ci en tant que réponses aux demandes des pilotes.\newline

			\item[Pilote : ]
			Fait office de \textbf{client}, envoie une demande d'information ATIS à la tour de contrôle, récupère cette information et replace une demande si cette information n'est pas intègre.\newline

			\item[Le centre météo : ]
			Programme \textbf{tiers} dans l'application, celui-ci se charge de générer automatiquement des informations ATIS différentes tout au long de son fonctionnement.\newline
		\end{description}

	\section{Rappel de l'énoncé}

		Des pilotes qui souhaitent décoller d’un aéroport non contrôlé ont besoin, pour ce faire, de connaître les informations ATIS. Celles-ci sont accessibles via un serveur.
		Chaque pilote va envoyer au serveur une demande ATIS. Le serveur va lors répondre à cette demande en allant chercher les informations nécessaires dans un fichier ATIS.
		Le pilote va recevoir ces informations et il doit alors obligatoirement répondre au serveur en lui envoyant soit :\newline
		\begin{itemize}
			\item Un ACK OK qui signifie \og informations bien reçues \fg et provoque la fin de la communication
			\item Un NAK KO qui signifie \og informations mal comprises \fg et nécessite de renvoyer les informations.
		\end{itemize}

		Le serveur doit pouvoir gérer un nombre indéfini de pilotes (restons réalistes).
		Le fichier ATIS contenant les informations nécessaires aux pilotes doit être régulièrement mis à jour par le gestionnaire météo.

\chapter{Analyse}
 \section{Plan global de l'application}
 	Voici un schéma donnant un aperçu rapide des fichiers constituant l'application, et les interactions entre eux via des liens légendés : \newline

	\includegraphics[width=\linewidth, frame]{schemasProjet.png} \newline

	Nos trois programmes qui constituent le coeur de l'application se trouvent sur la ligne au {\textbf{milieu} du schéma, écrits dans les fichiers {\textbf{\color{red} server.c}}, {      \textbf{\color{red} pilot.c}} et {\textbf{\color{red} meteo.c}}.

 Ces programmes vont se servir d' \og outils \fg fonctionnalisés dans le fichier {\textbf{\color{red} tools.c}} et obtenir les librairies, déclarations des fonctions outils et constantes  via inclusion du fichier {\textbf{\color{black} global.h}}.

\section{meteo.c}

	Lors de la mise en route de l'application, nous commençons par lancer le serveur Meteo.
	Celui-ci possède en son sein un certain nombre d'informations ATIS différentes à utiliser.
	Le serveur Meteo fonctionne sur une succession de boucles, dans lesquelles il va :

	- Sélectionner aléatoirement une des informations ATIS qu'il possède en mémoire.

	- Créer un fichier de verrouillage nommé "lock", afin de spécifier à son environnement qu'il est en train d'opérer.

	- Le serveur ouvre ensuite le fichier "meteo.txt" qui va contenir l'information ATIS actuelle destinée aux autres programmes, si ce fichier n'existait pas, Meteo le crée.

	- Si l'ouverture est un succès, Meteo écrit l'ATIS généré précédemment sur le fichier, ceci en troncature de manière à élimier une autre information ATIS qui est le résultat d'une opération précédente. Cette écriture va durer une seconde, seconde définie par nos soins.

	- Si l'écriture s'est bien déroulée, Meteo supprime le fichier de verrouillage "lock" pour spécifier à l'environnement que l'écriture est terminée et que "meteo.txt" est accessible, et termine la boucle après un repos défini de deux secondes.

	\section{server.c}

Après l'ouverture du serveur Meteo, nous lançons la tour de contrôle(server.c) , c'est-à-dire le serveur qui prendra en charge les demandes des pilotes.
Ce serveur fait divers opérations au lancement, il va : 

- Créer les deux fichiers fifo : watchtower\_send.fifo (fifo Output) et watchtower\_listen.fifo (fifo Input).

- Ouvrir ensuite les deux fichiers fifo.

- Si la création et l'ouverture des fifo est un succès, le serveur peut lancer les opérations. C'est-à-dire prendre en charge les différents pilotes grâce à une boucle.
Chaque seconde le serveur lit le fichier fifo Input où les pilotes écrivent leur demande. Si il y a une demande ATIS d'un pilote, le serveur lit le fichier meteo.txt qui contient l'ATIS correspondant. Il écrit ensuite cet ATIS sur le fichier fifo Output à l'attention du pilote. Si le fichier meteo.txt est occupé par le serveur Meteo ou qu'il n'existe pas, le serveur renverra un message d'erreur correspondant au pilote qui pourra reformuler sa demande.

	\section{pilot.c}
		
	Quand le serveur Meteo et le serveur de la tour de contrôle sont lancés, nous exécutons un certain nombre de pilote qui feront leur demande ATIS à la tour de contrôle.
	Chaque pilote va : 
	
	- Vérifier si le serveur et l'écriture du fifo Input est accessible.
	
	- Après la vérification, écrire sa demande ATIS sur le fifo Input à l'attention de la tour de contrôle.
	
	- Attendre ensuite la réponse du serveur en lisant le fichier fifo Output chaque seconde. Si la réponse est correcte, le pilote envoie un message (ACK) au serveur de confirmation de réception et ensuite "s'auto-detruit".  Si la réponse du serveur n'est pas correcte, le pilote envoie un message (NAK) pour demander au serveur de renvoyer l'ATIS et continue d'attendre la réponse.
		
\chapter{Détails des élements techniques spécifiques}
\chapter{Conclusion}

\begin{appendices}

\chapter{Code Source}
	\section{global.h}
		\lstinputlisting[language=C]{../global.h}
		\clearpage
	\section{meteo.c}
		\lstinputlisting[language=C]{../meteo.c}
		\clearpage
	\section{server.c}
		\lstinputlisting[language=C]{../server.c}
		\clearpage
	\section{pilot.c}
		\lstinputlisting[language=C]{../pilot.c}
		\clearpage
	\section{tools.c}
	  \lstinputlisting[language=C]{../tools.c}
	  \clearpage
\end{appendices}

\tableofcontents

\end{document}